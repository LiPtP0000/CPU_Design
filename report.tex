%!TEX program = xelatex
% 完整编译: xelatex -> biber/bibtex -> xelatex -> xelatex
\documentclass[lang=cn,a4paper,newtx]{elegantpaper}
\usepackage{algorithm}
\usepackage{algorithmicx}
\usepackage{algpseudocode}
\usepackage{subfig}
\usepackage{longtable}
\usepackage{lipsum}
\usepackage{makecell}
\usepackage{siunitx}
% \usepackage[backend=biber, style=gb7714-2015]{biblatex}
% \usepackage[title]{appendix}
% \usepackage{subfigure}
% \usepackage{subfloat}
\renewcommand{\listtablename}{表格目录}
\renewcommand{\appendixname}{附录~\Alph{section}}

\lstdefinelanguage{Assembly}{
  morekeywords={ ADD, SUB, MPY, JGZ, JMP, AND, OR,BUT,LOAD,STORE,SHIFTL,SHIFTR,HALT},  % 你可以在这里添加更多汇编指令
  sensitive=true,  % 保证区分大小写
  morecomment=[l]; % 单行注释(以分号开始)
  morestring=[b]",  % 字符串用双引号括起来
}
% Define the style for listings
\lstset{
  language=Verilog,  % 选择使用的语言
  basicstyle=\ttfamily\small,  % 基础字体风格
  keywordstyle=\color{blue}\bfseries,  % 指令的颜色和加粗
  commentstyle=\color{gray},  % 注释的颜色
  stringstyle=\color{red},  % 字符串的颜色
  identifierstyle=\color{purple},  % 标识符的颜色
  backgroundcolor=\color{lightgray!10}, % 背景色
  numbers=left,  % 行号显示在左侧
  stepnumber=1,  % 每行显示一个行号
  numberstyle=\tiny\color{gray},  % 行号的字体样式
  numbersep=5pt,  % 行号与代码之间的间距
  breaklines=true,  % 自动换行
  showstringspaces=false,  % 不显示字符串中的空格
  columns=flexible,  % 调整列宽
  frame=single,  % 在代码块外部加一个框
  framerule=0.5mm,  % 框线宽度
  rulesepcolor=\color{black},  % 分隔线颜色
  captionpos=b,  % 标题位置(b: bottom)
}
% \renewcommand{\refname}{参考文献}
\title{计算机组织与结构II:CPU设计文档}
\author{李勃璘 \\ 吴健雄学院}

\version{1.1}
\date{\zhdate{2025/4/11}}

% 本文档命令
\usepackage{array}
\newcommand{\ccr}[1]{\makecell{{\color{#1}\rule{1cm}{1cm}}}}
\addbibresource[location=local]{reference.bib} % 参考文献,不要删除

\begin{document}

\maketitle
\thispagestyle{empty}
\begin{abstract}
本设计文档详细阐述了一款基于 五级流水线 的 CPU 体系结构及其 Verilog 实现。CPU 采用 取指(IF)、译码(ID)、执行(EX)、访存(MEM)、写回(WB) 五级流水线,以提高指令执行效率。文档首先介绍了 CPU 的总体架构,包括 时钟与复位机制、关键存储单元、内部数据通路与控制信号,并详细说明了 指令集架构 及其编码格式。随后,针对流水线执行过程中可能出现的 数据冒险、控制冒险、流水线暂停 等问题,提出了 旁路(Forwarding)、分支预测(Branch Prediction)、冒险检测(Hazard Detection) 等优化方案,并给出了相应的 Verilog 设计。最后,文档分析了 CPU 与内存、总线及外部控制信号 的交互方式,并探讨了 Verilog 在 FPGA 上的实现方案。该设计通过流水线优化与高效控制信号管理,提升了 CPU 的吞吐率,为后续硬件优化和扩展提供了良好的基础。
\end{abstract}

% Removed
% \vspace{1cm}

% \textbf{版本更新记录:}

% \begin{longtable*}{|c|c|p{10cm}|}
%   \hline
%   \textbf{版本号} & \textbf{日期} & \textbf{更新内容} \\
%   \hline
%   \endfirsthead

%   \hline
%   \textbf{版本号} & \textbf{日期} & \textbf{更新内容} \\
%   \hline
%   \endhead

%   v1.0 & 2025-03-22 & 初始版本,包含基本 CPU 设计框架,流水线结构,Verilog 实现。 \\
%   \hline
%   v1.1 & 2025-03-29 & 修改 \\

% \end{longtable*}


\newpage
\pagenumbering{roman}
\tableofcontents
\newpage
\listoftables
\newpage
\pagenumbering{arabic}
\section{概述}
\lipsum[1]\footnote{手搓CPU是人类文明的伟大工程 —— 沃兹基硕德}


\section{体系结构设计}
\subsection{总体架构}
CPU由控制单元(CU),逻辑运算单元(ALU),内存(Memory)和寄存器组(Registers)组成,除内存以外,其余单元由被CU生成的控制信号控制的数据通路(Data Path)连接。另外,MAR和MBR分别还和地址总线、数据总线相连接,用于与内存交互。控制单元和内存都和控制总线相连接,用于与外部控制信号交互。

为简单起见,CPU的计算全部为\textbf{16位定点有符号数计算}。

\textcolor{blue}{这里需要一张图!!!}




\subsection{指令集架构}
指令集是指CPU能够对数据进行的所有操作的集合。每一条指令都可以被解释为寄存器与寄存器、内存、I/O端口之间的交互。交互方式由CU中的微指令(Micro-operation)给出,且每一条微指令都需要一个时钟执行(如不进行优化)。
\subsubsection{位宽设计}
地址段长为\textbf{8}位,指令码(Opcode)宽度为\textbf{8}位。因此,每一条指令的位宽为\textbf{16}位。
\subsubsection{寻址方式}
寻址方式指对地址段数据的解释方式。寻址方式由对应指令指定,支持表~\ref{tab:ISA:addressingmode}~中的全部寻址方式。由于给定的指令集高四位均空闲,使用最高位存储支持的寻址方式。
\begin{longtable}{c c c}
  \caption{指令集支持的寻址方式} \label{tab:ISA:addressingmode} \\
  \toprule
  寻址方式  & 描述 & 最高位\\
  \midrule
  \endfirsthead
  
  \caption[]{(续表)指令集支持的寻址方式} \\
  \toprule
  寻址方式  & 描述 & 最高位\\
  \midrule
  \endhead
  
  \midrule
  \multicolumn{3}{r}{续下页} \\
  \midrule
  \endfoot
  
  \bottomrule
  \endlastfoot
  
  立即数寻址   &  地址字段是操作数本身,数据为补码格式  & 0\\
  直接寻址 &  地址字段为存放操作数的地址    & 1\\
\end{longtable}

\subsubsection{指令集支持的指令}
指令集共支持13条不同的指令,列于表~\ref{tab:ISA:instructions}。每一条指令包含一个指令码,使用16进制格式存储。指令码的最高位为1时,寻址方式为立即数寻址;指令码的最高位为0时,寻址方式为直接寻址。
\begin{remark}
  指令中含*的仅支持直接寻址,因为立即数寻址对这些指令无意义。含$^\circledast$的仅支持立即数寻址。
\end{remark}

\begin{longtable}{c c c}
  \caption{指令集包含指令及功能(直接寻址下)} \label{tab:ISA:instructions} \\
  \toprule
  助记符  & 指令码(低四位) & 描述 \\
  \midrule
  \endfirsthead
  
  \caption[]{(续表)指令集包含指令及功能(直接寻址下)} \\
  \toprule
  助记符  & 指令码(低四位) & 描述 \\
  \midrule
  \endhead
  
  \midrule
  \multicolumn{3}{r}{续下页} \\
  \midrule
  \endfoot
  
  \bottomrule
  \endlastfoot
  
  *STORE X &  0001   & [X] $\leftarrow$ ACC  \\
  *LOAD X  & 0010    & ACC $\leftarrow$ [X]  \\
  ADD X   & 0011    & ACC $\leftarrow$ ACC + [X]\\
  SUB X   & 0100  & ACC $\leftarrow$ ACC - [X]\\
  JGZ X   & 0101    & ZF 和 NF 均为 0   ? PC $\leftarrow$ X : PC $\leftarrow$ PC + 1\\
  JMP X   & 0110    & PC $\leftarrow$ X\\
  *HALT    & 0111    & Stop\\
  MPY X   & 1000    & MR, ACC $\leftarrow$ ACC * [X],MR用于存储高16位 \\
  AND X   & 1001    & ACC $\leftarrow$ ACC \& [X]\\
  OR X    & 1010    & ACC $\leftarrow$ ACC | [X]\\
  NOT X   & 1011    & ACC $\leftarrow$ \~[X] \\
  $^\circledast$ SHIFTR  & 1100    & ACC $\leftarrow$ ACC $>>>$ 1,算术右移\\
  $^\circledast$ SHIFTL  & 1101    & ACC $\leftarrow$ ACC $<<<$ 1,算术左移\\
\end{longtable}

\subsection{CPU内部寄存器}
该部分描述CPU内部寄存器的含义、存储格式和数据被解释为的格式。这些寄存器通过CPU的内部数据通路相连接。寄存器操作是CPU快速操作的核心。

\begin{longtable}{c c c c c c}
  \caption{CPU内部寄存器的含义、总存储条数、单位位宽和数据解释格式} \label{tab:CPU:datawidth} \\
  \toprule
  寄存器 & 含义 & 条数 & 位宽 & 数据解释格式 & 归属模块\\
  \midrule
  \endfirsthead

  \caption[]{(续表)CPU内部寄存器的含义、总存储条数、单位位宽和数据解释格式} \\
  \toprule
  寄存器 & 含义 & 条数 & 位宽 & 数据解释格式 & 归属模块\\
  \midrule
  \endhead

  \midrule
  \multicolumn{6}{r}{续下页} \\
  \midrule
  \endfoot

  \bottomrule
  \endlastfoot

  PC   & 程序计数器,存储当前指令地址             & 1  & 8   & 指令码(Opcode) & /\\
  MAR  & 内存地址寄存器,存储要访问的内存地址     & 1  & 8   & 地址码(Address)& /\\
  MBR  & 内存缓冲寄存器,存储从内存读取或写入的数据 & 1  & 16  & 二进制补码 & /\\
  IR   & 指令寄存器,存储当前正在执行的指令       & 1  & 8   & 指令码(Opcode)& /\\
  % BR   & 通用寄存器,存储 ALU 计算中间结果        & 1  & 16  & 二进制补码 & ALU\\
  ACC  & 累加寄存器,存储 ALU 运算结果           & 1  & 16  & 二进制补码 & ALU\\
  MR   & 乘法寄存器,存储 ALU 乘法高 16 位       & 1  & 16  & 二进制补码 & ALU\\
  CM   & 控制存储器,存储微指令控制信号         & 未定 & 14  & 控制信号 & CU\\
  CAR  & 控制地址寄存器,指向当前执行的微指令   & 1  & 4   & CM中的条数下标 & CU\\
  CBR  & 控制缓冲寄存器,存储当前微指令的控制信号 & 1  & 14  & 控制信号 & CU\\
\end{longtable}

除上述寄存器以外,ALU进行运算时还会更改\textbf{状态寄存器}(Flags),用于CU进行条件判断。例如,JGZ命令需要判断上一步的运算结果是否是0,CU可以直接通过状态寄存器中的ZF(Zero Flag)寄存器直接进行判断。本设计中使用的所有状态寄存器见表~\ref{tab:CPU:status},它们都直接连向CU,通路不受控制信号的控制。

\begin{longtable}{c c c}
  \caption{状态寄存器列表} \label{tab:CPU:status} \\
  \toprule
  寄存器 & 全称 & 行为 \\ 
  \midrule
  \endfirsthead

  \caption[]{(续表)状态寄存器列表} \\
  \toprule
  寄存器 & 全称 & 行为\\
  \midrule
  \endhead

  \midrule
  \multicolumn{3}{r}{续下页} \\
  \midrule
  \endfoot

  \bottomrule
  \endlastfoot

  ZF   & Zero Flag             & ALU运算结果(通常为ACC)为0时置1\\
  CF  & Carry Flag     & 存储算术移位移出的比特(由于有符号数不存储进位)\\
  OF  & Overflow Flag &  ALU运算结果发生溢出时置1\\
  NF  & Negative Flag &  ALU运算结果为负数时置1\\
  MF & Multiply Flag & ALU此轮运算出现乘法,且使用MR寄存器时置1\\
\end{longtable}
\subsection{算术逻辑单元ALU}
算术逻辑单元ALU负责进行大部分CPU内的计算\footnote{PC自增与PC赋值在设计中不引入ALU。}。它通过MBR获取运算的第二个数据,计算方式受状态寄存器影响,计算结果存入ACC寄存器中(若有乘法则可能存入MR寄存器)。

\begin{longtable}{c c c c c c c}
  \caption{各指令对标志位的影响} \label{tab:ISA:flags} \\
  \toprule
  操作码低四位 & 助记符 & ZF & CF & OF & NF & MF\\
  \midrule
  \endfirsthead
  
  \caption[]{(续表)各指令对标志位的影响} \\
  \toprule
  操作码低四位 & 助记符 & ZF & CF & OF & NF & MF\\
  \midrule
  \endhead
  
  \midrule
  \multicolumn{7}{r}{续下页} \\
  \midrule
  \endfoot
  
  \bottomrule
  \endlastfoot
  
  0001 & ADD     & 为0时置1 & 0 & 溢出置1 & 负数置1 & 0 \\
  0010 & SUB     & 为0时置1 & 0 & 溢出置1 & 负数置1 & 0 \\
  0011 & MPY     & 为0时置1 & 0 & 0 & 负数置1 & MR不为0时置1 \\
  0100 & AND     & 为0时置1 & 0 & 0 & 负数置1 & 0 \\
  0101 & OR      & 为0时置1 & 0 & 0 & 负数置1 & 0 \\
  0110 & NOT     & 为0时置1 & 0 & 0 & 负数置1 & 0 \\
  0111 & SHIFTR  & 为0时置1 & 最低位 & 0 & 负数置1 & 0 \\
  1000 & SHIFTL  & 为0时置1 & 最高位 & 0 & 负数置1 & 0 \\

\end{longtable}

\subsubsection{ALU的数据通路}
\begin{longtable}{c c c}
  \caption{ALU的数据通路} \label{tab:ALU:DataPath} \\
  \toprule
  源寄存器 & 目标寄存器 & 传递的数据 \\ 
  \midrule
  \endfirsthead

  \caption[]{(续表)ALU的数据通路} \\
  \toprule
  源寄存器 & 目标寄存器 & 传递的数据\\
  \midrule
  \endhead

  \midrule
  \multicolumn{3}{r}{续下页} \\
  \midrule
  \endfoot

  \bottomrule
  \endlastfoot

     &              & \\
    &      & \\
    & &  \\
    &  & \\
\end{longtable}
\subsubsection{ALU的内部控制信号}
ALU与外围寄存器的控制通路见第~\ref{sec:datapath}~节。为了让ALU内部获取要执行的运算操作,需要在ALU内部根据IR提供的指令类型翻译为ALU的控制信号。在本设计中,每一种运算都对应一种控制信号。

% 放一张表

\subsubsection{ALU的内部控制逻辑}
这一部分描述ALU的内部控制信号与来自状态寄存器的Flags如何控制ALU内部数据通路的开关。

% 放一张表

\subsection{CPU内部数据通路、控制信号与微操作指令(Micro-Operations)}
CPU中总线、寄存器、内存和ALU等关键器件需要以合理的时序执行不同的指令,而这需要数据通路和控制通路共同实现。数据通路负责在各个器件中传递数据,而控制通路负责控制数据通路的开关。因此,在本节中,先定义CPU需要的数据通路,再进一步定义控制数据通路的控制信号,最后描述承载控制信号的微指令。

\subsubsection{数据通路与控制信号}\label{sec:datapath}
关键存储单元之间通过数据通路进行连接。每条数据通路都由一位控制信号控制。控制信号为1时表示通路打开,数据沿指定流向进行传输。该CPU共有\textbf{14}位控制信号。
\begin{remark}
  CU内部的数据通路不需要受到微指令的控制,因此不需要在此列出。请参照模块设计部分。
\end{remark}
\begin{longtable}{c c c}
  \caption{数据通路与控制信号一览} \label{tab:CPU:DataPath} \\
  \toprule
  控制信号位 & 源寄存器/单元  & 目的寄存器/单元  \\
  \midrule
  \endfirsthead

  \caption[]{(续表)数据通路与控制信号一览} \\
  \toprule
  控制信号位 & 源寄存器/单元  & 目的寄存器/单元  \\
  \midrule
  \endhead

  \midrule
  \multicolumn{3}{r}{续下页} \\
  \midrule
  \endfoot

  \bottomrule
  \endlastfoot

  0  & MAR   & 地址总线  \\
  1  & PC    & MBR  \\
  2  & PC    & MAR  \\
  3  & MBR   & PC  \\
  4  & MBR   & IR  \\
  5  & 数据总线 & MBR  \\
  6  & MBR   & ALU(待处理数据) \\
  7  & ACC和MR   & ALU  \\
  8  & MBR   & MAR  \\
  9  & ALU   & ACC  \\
  10 & MBR   & ACC  \\
  11 & ACC   & MBR  \\
  12 & MBR   & 数据总线  \\
  13 & IR    & CU  \\
  \hline
  ALU内部控制通路 & & \\
  \hline
  14 & Why am I doing this & Confusing \\
\end{longtable}

\subsubsection{微操作指令(Micro-Operations)}
为了实现指令集中所有指令,需要将指令集的指令分解为多步微操作指令。为了和后续流水线部分对齐,在微操作指令的设计上,也分为和六级流水线相同的步骤。

在本设计中,采用水平微指令(Horizontal Micro-operation)设计。这意味着每一个微操作指令携带所有控制信号位的开关和下一个微操作指令的地址。

所有用逗号隔开的指令均\textbf{从左到右顺序执行},不能并行执行。所有的微操作指令见表~\ref{tab:microops}。
% \begin{longtable}{c c c c}
%   \caption{CISC 指令拆分、微操作与控制信号一览} \label{tab:CISC2RISC} \\
%   \toprule
%   指令 & 拆分的 RISC 指令 & 微操作(五级流水) & 控制信号位 \\
%   \midrule
%   \endfirsthead

%   \caption[]{(续表)CISC 指令拆分、微操作与控制信号一览} \\
%   \toprule
%   指令 & 拆分的 RISC 指令 & 微操作(五级流水) & 控制信号位 \\
%   \midrule
%   \endhead

%   \midrule
%   \multicolumn{4}{r}{续下页} \\
%   \midrule
%   \endfoot

%   \bottomrule
%   \endlastfoot

% ADD X & LOAD X; ADD MBR & 
% \begin{tabular}{l}
% IF: PC→MAR, Mem[MAR]→MBR \\
% ID: MBR→IR, IR→Decode \\
% EX1: X→MAR (C6) \\
% MEM1: Mem[MAR]→MBR (C5) \\
% EX2: ACC+MBR→ALU (C7,C9) \\
% WB: ALU→ACC (C9)
% \end{tabular}
% & C6, C5, C7, C9 \\

% SUB X & LOAD X; SUB MBR & 同上, EX2为ACC-MBR & C6, C5, C7, C9 \\

% MPY X & LOAD X; MPY MBR & 同上, EX2为ACC×MBR & C6, C5, C7, C9 \\

% STORE X & STORE ACC → [X] & 
% \begin{tabular}{l}
% IF-ID-EX: X→MAR (C6) \\
% MEM: ACC→MBR (C8) \\
% WB: MBR→Mem[MAR] (C5)
% \end{tabular}
% & C6, C8, C5 \\

% LOAD X & LOAD Mem[X] → ACC &
% \begin{tabular}{l}
% IF-ID-EX: X→MAR (C6) \\
% MEM: Mem[MAR]→MBR (C5) \\
% WB: MBR→ACC (C9)
% \end{tabular}
% & C6, C5, C9 \\

% \end{longtable}

\begin{longtable}{cccc}
  \caption{五级流水线控制信号表} \label{tab:compact_pipeline}\\
  \toprule
  指令 & \makecell{EX阶段\\(微操作/控制信号)} & \makecell{MEM阶段\\(微操作/控制信号)} & \makecell{WB阶段\\(微操作/控制信号)} \\
  \midrule
  \endfirsthead

  \toprule
  \caption[]{(续表)五级流水线控制信号表} \\
  \toprule
  指令 & EX阶段 & MEM阶段 & WB阶段 \\
  \midrule
  \endhead

  \midrule
  \multicolumn{4}{r}{续下页} \\
  \midrule
  \endfoot

  \bottomrule
  \endlastfoot

  %%% 通用阶段 %%%
  \multicolumn{4}{l}{所有指令共享的公共阶段:} \\
  IF & \multicolumn{3}{l}{MAR←PC; PC←PC+1; MBR←Mem[MAR] \quad (C0,C2,C3,C5)} \\
  ID & \multicolumn{3}{l}{IR←MBR; CU←IR \quad (C4,C13)} \\
  \midrule

  %%% 存储指令 %%%
  STORE X & 
  MAR←X \quad (C6) & 
  Mem[MAR]←ACC \quad (C11,C12) & 
  无操作 \\
  \midrule

  %%% 加载指令 %%%
  LOAD X & 
  MAR←X \quad (C6) & 
  MBR←Mem[MAR] \quad (C5) & 
  ACC←MBR \quad (C10) \\
  \midrule

  %%% 算术运算 %%%
  ADD X & 
  MAR←X \quad (C6) & 
  MBR←Mem[MAR] \quad (C5) & 
  \makecell{ACC←ACC+MBR;\\ \quad (C7,C9)} \\
  
  SUB X & 
  MAR←X \quad (C6) & 
  MBR←Mem[MAR] \quad (C5) & 
  \makecell{ACC←ACC-MBR;\\ \quad (C7,C9)} \\
  
  MPY X & 
  MAR←X \quad (C6) & 
  MBR←Mem[MAR] \quad (C5) & 
  \makecell{ACC,MR←ACC×MBR;\\ \quad (C7,C9,C14)} \\
  \midrule

  %%% 逻辑运算 %%%
  AND X & 
  MAR←X \quad (C6) & 
  MBR←Mem[MAR] \quad (C5) & 
  \makecell{ACC←ACC AND MBR;\\ \quad (C7,C9)} \\
  
  OR X & 
  MAR←X \quad (C6) & 
  MBR←Mem[MAR] \quad (C5) & 
  \makecell{ACC←ACC OR MBR; \\\quad (C7,C9)} \\
  
  NOT X & 
  MAR←X \quad (C6) & 
  MBR←Mem[MAR] \quad (C5) & 
  \makecell{ACC←NOT MBR;\\ \quad (C7,C9)} \\
  \midrule

  %%% 移位操作 %%%
  SHIFTR & 
  准备移位量 \quad (C8) & 
  无操作 & 
  ACC←ACC$\ggg$1;\quad (C7,C9) \\
  
  SHIFTL & 
  准备移位量 \quad (C8) & 
  无操作 & 
  ACC←ACC$\lll$1;  \quad (C7,C9) \\
  \midrule

  %%% 控制指令 %%%
  JMP X & 
  PC←X \quad (C6) & 
  流水线清空 & 
  无操作 \\
  
  JGZ X & 
  检查条件(ZF=0,NF=0) & 
  若条件成立:PC←X \quad (C6) & 
  流水线清空 \\
  
  HALT & 
  \texttt{enable}←0 & 
  冻结流水线 & 
  无操作 \\
\end{longtable}





\subsection{内存(RAM)}
内存(RAM)存储指令集和CPU保存的数据。内存的大小为 512 Byte,每条存入内存的数据位宽为16,共能存入256条数据。

内存支持总线读写,并受三条总线控制:地址总线、数据总线和控制总线。\textbf{控制总线}中的外部控制信号决定在这个周期中内存的读/写状态,是否向数据总线写入,同步时序等功能。CPU与内存(RAM)通过两条总线交互,分别为\textbf{地址总线}和\textbf{数据总线}。内存通过读取地址总线决定写入内存中的地址,通过读取数据总线决定写入指定地址中的数据。关于总线的具体配置见~\ref{sec:ExternalControl}。

采用Harvard结构,内存中\textcolor{red}{不存放待执行指令}。待执行指令存放于另一块内存中,防止数据通路和指令通路发生冲突。
\subsection{总线与外部控制信号}\label{sec:ExternalControl}

\subsubsection{地址总线}
地址总线为\textbf{8}位单向总线,提供CPU(即MAR)到内存的地址传送通路。由于其为单向总线,仅需内存侧读使能信号与CPU侧MAR的控制信号控制即可,无需复用。
\subsubsection{数据总线}
数据总线为\textbf{16}位双向总线,提供CPU(即MBR)与内存的双向数据通路。数据总线采用分时复用的方式进行设计。\textcolor{red}{如果MEM和WB在同一时钟下,会出现流水线冲突,需要考虑是否引入旁路。}
\subsubsection{控制总线与外部控制信号(需要等待流水线设计好)}
外部控制信号是一组单比特信号,通过控制总线控制CU和RAM的行为,它们受流水线周期的控制置0或置1。CU和RAM通过内部映射决定监视哪些位的信号。外部控制信号主要包括以下功能:
\begin{itemize}
  \item RAM读写控制;
  \item 流水线控制信号。
\end{itemize}

所有外部控制信号列于表~\ref{tab:CPU:ExternalControl}。
\begin{longtable}{c c c c c}
  \caption{外部控制信号一览} \label{tab:CPU:ExternalControl} \\
  \toprule
  控制信号位/类型 & 别名  & 有效模块 & 高电平时作用 & 低电平时作用\\
  \midrule
  \endfirsthead

  \caption[]{(续表)外部控制信号一览} \\
  \toprule
  控制信号位/类型 & 别名  & 有效模块 & 高电平时作用 & 低电平时作用 \\
  \midrule
  \endhead

  \midrule
  \multicolumn{5}{r}{续下页} \\
  \midrule
  \endfoot

  \bottomrule
  \endlastfoot
  \textbf{RAM读写控制}\\
  \hline
  0  & MemoryWrite    & RAM  & RAM写数据总线 & 无\\
  1  & MemoryRead     & RAM  & RAM读数据总线和地址总线 & 无\\
  \hline
  \textbf{分支预测} \\
  \hline
  2   & BranchTaken    & CU   & 执行Branch     & 顺序执行 \\
  3   & Jump            & CU  & 执行Jump        & 顺序执行 \\
  \hline
  \textbf{流水线控制}  \\
  \hline
  4   & PipelineStall & CU    & 流水线暂停(如数据冒险、存储器访问延迟) & 无 \\
  5   & PipelineFlush & CU    & 流水线清空(如错误预测、异常发生)      & 无 \\

\end{longtable}
\section{I/O接口}
\subsection{用户端代码解释}
目前采用\textbf{用户编写汇编代码}$\to$\textbf{转换为16位机器码}的方式输入指令。用户可在文本编辑器中编写类汇编代码,而解释器负责将其解释为机器码。

以从1加到100的程序举例:
\begin{lstlisting}[language=Assembly]
  LOAD IMMEDIATE 0    ; 初始化累加器为0 → ACC=0
  STORE 1            ; 存储到地址1(SUM变量)
  LOAD IMMEDIATE 1    ; 初始化计数器为1 → ACC=1
  STORE 2            ; 存储到地址2(i计数器)
LOOP: LOAD 0             ; 读取当前累加值 → ACC=SUM
  ADD 1              ; 加上当前计数器值 → ACC=SUM+i
  STORE 1            ; 更新累加值 → SUM=SUM+i
  LOAD 2             ; 读取计数器 → ACC=i
  ADD IMMEDIATE 1    ; 计数器自增 → ACC=i+1
  STORE 2            ; 更新计数器 → i=i+1
  SUB IMMEDIATE 100  ; 比较是否达到100 → ACC=i-100
  JGZ LOOP           ; 如果i<=100(即ACC<=0),继续循环
  HALT
\end{lstlisting}

代码最终将被解释为一串二进制比特流,解释服从:
\begin{itemize}
  \item 地址占1byte,Opcode占1byte;
  \item 含IMMEDIATE关键字的行,Opcode的MSB为1;
  \item 在代码解释的过程中,\texttt{LOOP}应映射到相同行指令的地址。
\end{itemize}

拟采用Python语言对用户代码进行处理,处理好的比特流可通过UART串口通信传入FPGA的BRAM。


\subsection{UART 接收与 指令RAM 写入设计}

本设计基于 NEXYS 4 DDR 开发板,通过其板载的 UART 接口完成主机与 FPGA 之间的数据传输。该方案无需额外的数据线或 I/O 资源,即可实现对 FPGA 内部 RAM 的程序写入与指令输入,提升了系统的硬件集成度与使用便捷性。

\subsubsection{UART 接收逻辑}

开发板主系统时钟频率为 \SI{100}{\mega\hertz},串口通信波特率设定为 \SI{115200}{bps}。根据 UART 通信协议,每接收 $1$ 位数据所需的时钟周期数为:

\begin{equation}
  \text{CLK\_BAUD} = \frac{\text{CLK\_FREQ}}{\text{BAUD\_RATE}} = \frac{100\,000\,000}{115200} \approx 868
\end{equation}

采用常见的 \textbf{8N1} 格式传输,即每帧包括:

\begin{itemize}
  \item 1 位起始位(Start Bit);
  \item 8 位数据位(Data Bits);
  \item 1 位停止位(Stop Bit)。
\end{itemize}

因此,每帧共 $10$ 位,总计需要约:

\begin{equation}
  \text{CLK\_FRAME} = 10 \times \text{CLK\_BAUD} = 10 \times 868 = 8680\ \text{cycles}
\end{equation}

接收端采用中点采样策略,即在每位传输中间时刻(约第 $434$ 个时钟周期)对数据位进行采样,以提升抗干扰能力。

\subsubsection{数据缓冲与存储结构}

为保证串口数据完整接收,接收模块首先将每帧数据写入异步 FIFO 缓冲区。随后由控制逻辑从 FIFO 中读取数据,并写入开发板内部的块 RAM(通过 Vivado IP 核生成)。

\paragraph{数据写入格式:}

\begin{itemize}
  \item RAM 的每个地址对应两个字节(16 位)数据;
  \item 高字节为操作码(Opcode),低字节为立即数或地址(Operand);
  \item 若当前行含有 \texttt{IMMEDIATE} 关键字,则 Opcode 的最高位(MSB)为 $1$。
\end{itemize}

\paragraph{写入控制规则:}

\begin{itemize}
  \item 每一条指令都为2byte指令,即使其可能没有操作数。对于没有操作数的情况,将操作数位置补零。
  \item RAM 地址从地址 0 开始顺序写入。
  \item 程序中如含有 \texttt{LOOP} 标签,将在 RAM 地址分配完毕后由软件在解析阶段回填其地址位置。
\end{itemize}

另外,传入FPGA的所有指令将存于单独的指令内存中,与CPU数据内存隔离开来。CPU内存仅存数据,这符合用户编写的直观感受。每条存入内存的数据位宽为16,即每个地址按顺序存放一条指令。地址从1开始依次递增,防止复位时地址位初始化为0导致出错。指令写入在CPU开机之前,写入成功后开发板亮蓝灯。若在CPU运行时没有指令,则开发板亮红灯。

本模块通过串口实现了简洁的二进制指令数据装载方式,降低了外设复杂性,为后续的控制单元译码与执行单元操作提供了明确的数据支持。

\subsection{用户交互设计}
该部分描述用户与FPGA的交互接口(按钮、按键等)以及看到的运行状态信息与结果显示设计。
\section{流水线架构与优化策略}
\subsection{总体架构}
为了加速CPU的指令执行速度,采用\textbf{5级同步流水线}完成CPU执行指令的全流程。分别为:取指令(IF),指令译码(ID),指令执行(EX),内存访问(MEM)和写回寄存器(WB)五个阶段。流水线的五个阶段如下所示。
$$
  \text{IF} \rightarrow \text{ID} \rightarrow \text{EX} \rightarrow \text{MEM} \rightarrow \text{WB}
$$

流水线各阶段的主要工作如下:
\begin{itemize}
  \item \textbf{IF(Instruction Fetch):}从指令存储器中取出指令,同时确定下一条指令地址(指针指向下一条指令);
  \item \textbf{ID(Instruction Decode):}翻译指令,同时让计算机得出要使用的寄存器,或者让立即数进行拓展(方便后续指令执行),亦或者(转移指令)是给出转移目的寄存器与转移条件;
  \item \textbf{EX(Execution):}按照微操作指令指示打开数据通路。
  \item \textbf{MEM(Memory):}若为\texttt{LOAD/STORE}指令,这个阶段就要访问存储器。此外,指令从EX向下执行到WB阶段。另外,在这个阶段还要判定是否有异常要处理,如果有,那么就清除流水线,然后转移到异常处理例程入口地址处继续执行。
  \item \textbf{WB(Write Back):}将运算结果保存到目标寄存器。
\end{itemize}

\subsection{流水线冒险}
包含三种冒险情况:结构冒险、数据冒险、控制冒险(分支冒险)。\cite{zhihu453232311}

结构冒险,即\textbf{某指令引用了前一次的运算结果,但其结果值还未产生}。对于这种情况,采用旁路(Forwarding)技术解决。
\subsection{流水线优化:分支预测}
在本设计中,采用1比特分支预测进行流水线优化。分支预测的控制信号由\textbf{控制总线}输入到CU中。分支预测的步骤如算法~\ref{alg:1bitBP}。

\begin{algorithm}[htbp]
  \caption{1-bit 分支预测}
  \label{alg:1bitBP}
  \begin{algorithmic}[1]
  \State \textbf{Note:} 使用 1-bit 预测器预测分支是否跳转
  \State \textbf{Input:} $PC, BranchTaken$
  \State \textbf{Output:} $NextPC$
  \State \textbf{Internal Registers:} $PredictorBit, BranchTarget$

  \Procedure{Branch Prediction}{}
      \If {$PredictorBit == 1$}
          \State $PredictedTaken \gets \text{True}$
          \State $NextPC \gets BranchTarget$
      \Else
          \State $PredictedTaken \gets \text{False}$
          \State $NextPC \gets PC + 1$
      \EndIf
  \EndProcedure

  \Procedure{Branch Resolution}{}
      \If {$BranchTaken \neq PredictedTaken$}  \Comment{预测错误}
          \State $Flush \gets 1$  \Comment{清空错误指令}
          \State $NextPC \gets \textbf{if } BranchTaken \textbf{ then } BranchTarget \textbf{ else } PC + 1$
      \EndIf
      \State \textbf{Update Predictor:}
      \State $PredictorBit \gets BranchTaken$  \Comment{更新 1-bit 预测器}
  \EndProcedure
  \end{algorithmic}
\end{algorithm}


% \subsubsection{流水线优化}\label{sec:streamline}
% 流水线设计下会出现内存读写冲突等,需要实现冲突现象的检测模块。同时,如果没有流水线优化策略,则每次访问冲突都需要清空流水线,这会降低流水线的运行效率。

\section{模块设计}
\subsection{时钟、复位与停止信号}
CPU由\textbf{全局同步时钟}控制,时钟主频为50MHz。除复位信号外,所有控制逻辑与计算逻辑全部在时钟上升沿进行。UART传输部分使用100MHz的时钟主频。

CPU设有\textbf{全局异步复位}信号,低电平有效。当异步复位时,内存中除指令集数据以外所有数据清空,所有寄存器清空,控制信号全部归为断开(0)。

当CPU执行07号指令HALT时,CPU处于\textbf{暂停}状态。与复位不同的是,此时所有寄存器不清空,但所有通路断开。在模块中使用\texttt{enable}信号标识(低电平有效)。恢复程序运行的方法是全局复位或继续运行信号(绑定FPGA的按键)。当该按键被按下时,\texttt{enable}信号恢复为1。
\subsection{UART传输与指令写入}
指令写入通过Python脚本(附录~\ref{sec:appendices:python}~)完成,其可以根据用户输出一串UART格式的比特流。用户可通过PC上的串口调试设备连接UART端口进行传输。

\subsubsection{UART模块}
\paragraph{模块基本信息:}
\begin{itemize}
  \item 模块名:\texttt{UART}
  \item 最新更新日期:\texttt{4.13}
  \item 是否经过测试:否
\end{itemize}
\paragraph{模块功能:}
将用户输入代码比特流(8N1格式)译码为1字节数据。
\paragraph{模块外部接口:}

\begin{longtable}{>{\bfseries}c c c c}
  \toprule
  信号名 & 方向 & 位宽 & 描述 \\
  \midrule
  \endhead
  
  \texttt{i\_clk}   & 输入  & 1      & 系统时钟信号 \\
  \texttt{i\_rst\_n} & 输入  & 1      & 全局复位信号 \\
  \texttt{i\_rx}    & 输入  & 1      & UART 接收引脚 \\
  \texttt{o\_data}  & 输出  & 8    & 接收到的一帧数据 \\
  \texttt{o\_valid} & 输出  & 1      & 数据有效标志,高电平表示 \texttt{o\_data} 已生成 \\
  \texttt{o\_clear\_sign}    & 输出   & 1       & 表示UART输入结束(第一次输入结束后0.5秒内无新的输入)\\
  \bottomrule
  \end{longtable}

\subsubsection{FIFO模块}
\paragraph{模块基本信息:}
\begin{itemize}
  \item 模块名:\texttt{FIFO}
  \item 最新更新日期:\texttt{4.13}
  \item 是否经过测试:否
\end{itemize}
\paragraph{模块功能:}
异步FIFO,缓存UART数据。将每两个读出的UART数据拼成2字节的指令输出给BRAM。
\paragraph{模块外部接口:}
\begin{longtable}{>{\bfseries}c c c c}
  \toprule
  信号名 & 方向 & 位宽 & 描述 \\
  \midrule
  \endhead
  
  \texttt{i\_rst\_n}         & 输入  & 1        & 异步复位信号,低有效 \\
  \texttt{i\_clk\_wr}        & 输入  & 1        & 写时钟信号,UART 使用的 100MHz 时钟 \\
  \texttt{i\_valid\_uart}    & 输入  & 1        & 表示当前 UART 输入数据有效 \\
  \texttt{i\_data\_uart}     & 输入  & 8        & UART 接收到的 8 位数据字节 \\
  \texttt{i\_clk\_rd}        & 输入  & 1        & 读时钟信号,主系统使用的 50MHz 时钟 \\
  \texttt{o\_data\_bram}     & 输出  & 16       & 两个 UART 字节拼接后的数据,写入 BRAM \\
  \texttt{o\_addr\_bram}     & 输出  & 8        & BRAM 写入地址,从 0 开始自增 \\
  \texttt{o\_wr\_en\_bram}   & 输出  & 1        & BRAM 写使能,高电平表示写入有效 \\
  \texttt{o\_fifo\_empty}    & 输出   & 1       & 表示FIFO空(作为输入完成的判据)\\
  \bottomrule
\end{longtable}

\subsubsection{指令BRAM模块}
\paragraph{模块基本信息:}
\begin{itemize}
  \item 模块名:\texttt{BRAM\_INSTR}
  \item 最新更新日期:\texttt{4.13}
  \item 是否经过测试:否
\end{itemize}
\paragraph{模块功能:}
描述一指令块RAM,可存放256条2byte指令。读写双口,拥有写使能(FIFO传入)。外部设备可通过地址读取对应地址的2byte指令。
\paragraph{模块外部接口:}
\begin{longtable}{>{\bfseries}c c c c}
  \toprule
  信号名 & 方向 & 位宽 & 描述 \\
  \midrule
  \endhead

  \texttt{i\_clk}          & 输入  & 1        & 系统时钟信号,驱动读写操作 \\
  \texttt{en\_write}       & 输入  & 1        & 写使能信号,高电平时允许将指令写入 BRAM \\
  \texttt{i\_addr\_write}  & 输入  & 8        & 要写入的指令地址 \\
  \texttt{i\_instr\_write} & 输入  & 16       & 要写入的指令内容 \\
  \texttt{i\_addr\_read}   & 输入  & 8        & 要读取的指令地址 \\
  \texttt{o\_instr\_read}  & 输出  & 16       & 从 BRAM 中读取的指令内容 \\
  \bottomrule
\end{longtable}

\subsection{ALU}
\subsubsection{ALU运算结果}
ALU运算结果存放于MR寄存器和ACC寄存器中。其中MR寄存器存放乘法运算的高位结果,ACC寄存器存放乘法运算的低位结果。ALU与两个寄存器的数据通路都由7号控制位控制。MR连向ALU的高位结果,ACC连向ALU的低位结果。当控制信号MF为1时,若需要访存,则添加一个MEM阶段进行MR的访存。

\subsection{CU}
参考表~\ref{tab:CPU:DataPath}~,确定指令集中的每一条指令对应的控制信号,并存储到CU的内部寄存器中,即可完成CU的主要功能设计。


\subsection{CPU内部寄存器}
\subsubsection{MAR}
\subsubsection{MBR}
\subsubsection{IR}
每条指令在ID阶段被传入IR寄存器。IR寄存器可通过判断最高位来确定寻址方式。即:
\begin{itemize}
  \item IR最高位为1,立即数寻址,在MEM阶段不再进行访存,而是直接传入MBR。
  \item IR最高位为0,直接寻址,在MEM阶段进行访存。
\end{itemize}

\subsubsection{PC}

\subsubsection{ACC}
\subsection{Memory}



% \subsection{分支预测}

\section{仿真验证}
\subsection{时延分析}
% 定义一个打印周期为从POC重置自身$SR_7$到一轮打印结束所经历的时钟上升沿个数。那么:
% \begin{enumerate}
%     \item 由POC重置到PROCESSOR置零$SR_7$,使用6个时钟,分别为读取POC信号的一个时钟和状态机的五个时钟;\footnote{中断模式较查询模式少用一个时钟。因为中断信号由组合逻辑生成,中断模式下PROCESSOR可立刻读取中断信号,无需读取POC信号。}
%     \item POC此时开始与打印机进行交互。写入\texttt{o\_tr}和\texttt{o\_pd}共使用1个时钟,\texttt{o\_tr}信号持续1个时钟;
%     \item 打印机收到信号后,耗时8个时钟打印数据;
%     \item 打印结束后,POC重置$SR_7$,使用1个时钟。
% \end{enumerate}
% 故一个打印周期为17个时钟(查询模式)或16个时钟(中断模式)。
\subsection{并行计算加速比(Speed-up Factor)分析}
\subsection{激励设置}

\section{FPGA实现}
\subsection{用户输入端}
采用第一个测试样例进行测试。
$$
1+2+\dots +99 +100 =5050
$$
编写源程序如下:
\begin{lstlisting}[language=Assembly]
      LOAD IMMEDIATE 0 
      STORE 1
      LOAD IMMEDIATE 1
      STORE 2
      LOAD IMMEDIATE 100
      STORE 3
LOOP: LOAD 1
      ADD 2
      STORE 1
      LOAD 2
      ADD IMMEDIATE 1
      STORE 2
      LOAD 2
      SUB 3
      JGZ LOOP

      HALT
\end{lstlisting}
\nocite{FPGA-CPU}

\printbibliography
\newpage
\addappheadtotoc
\begin{appendices}
  \section{完整设计代码}
  \subsection{汇编程序处理Python脚本}\label{sec:appendices:python}
  \begin{lstlisting}[language=python,caption={convert.py}]
import re

# memonics
MEMONICS = {
    "STORE":  0x01,
    "LOAD": 0x02,
    "ADD":   0x03,
    "SUB":   0x04,
    "JGZ":   0x05,
    "JMP":  0x06,
    "HALT":  0x07,
    "MPY":  0x08,
    "AND":  0x09,
    "OR":   0x10,
    "NOT":  0x11,
    "SHIFTR": 0x12,
    "SHIFTL":   0x13
}

def parse_assembly(lines):
    machine_code = []
    labels = {}
    pending = []

    # First pass: find labels
    addr = 0
    for line in lines:
        line = line.split(';')[0].strip()  # clear comments
        if not line:
            continue
        if ':' in line:
            label, rest = map(str.strip, line.split(':', 1))
            labels[label] = addr
            if rest:
                addr += 1
        else:
            addr += 1

    # Second pass: generate code
    addr = 0
    for line in lines:
        line = line.split(';')[0].strip()
        if not line:
            continue
        if ':' in line:
            parts = line.split(':', 1)
            line = parts[1].strip()
            if not line:
                continue

        tokens = line.split()
        if not tokens:
            continue

        instr = tokens[0]
        immediate = False

        if instr in ["HALT","SHIFTR","SHIFTL"] : # No operand, fill with 0
            opcode = MEMONICS[instr]
            operand = 0x00
        else:
            if len(tokens) < 2:
                raise ValueError(f"Missing operand in line: {line}")

            if tokens[1] == "IMMEDIATE":
                immediate = True
                operand_str = tokens[2]
                opcode = MEMONICS[instr] | 0x80  # MSB = 1
            else:
                operand_str = tokens[1]
                opcode = MEMONICS[instr]  # MSB = 0

            if operand_str.isdigit():
                operand = int(operand_str)
            elif operand_str in labels:
                operand = labels[operand_str]
            else:
                try:
                    operand = int(operand_str, 0)  # Support 0x form operand
                except:
                    raise ValueError(f"Unknown operand: {operand_str}")

        if operand < 0 or operand > 255:
            raise ValueError(f"Operand out of 8-bit range: {operand}")

        machine_code.append((opcode << 8) | operand)
        addr += 1

    return machine_code


def assemble_to_bytes(code: list[int]) -> bytearray:
    result = bytearray()
    for word in code:
        result.append((word >> 8) & 0xFF)  # opcode
        result.append(word & 0xFF)         # operand
    return result


if __name__ == "__main__":
    # 示例程序:从1加到100
    asm_code = """
      LOAD IMMEDIATE 0
      STORE 1
      LOAD IMMEDIATE 1
      STORE 2
    LOOP: LOAD 1
      ADD 2
      STORE 1
      LOAD 2
      ADD IMMEDIATE 1
      STORE 2
      SUB IMMEDIATE 100
      JGZ LOOP
      HALT
    """

    lines = asm_code.strip().split('\n')
    machine_words = parse_assembly(lines)
    binary = assemble_to_bytes(machine_words)

    # 打印每条机器码(16位)和最终二进制流
    print("机器码:")
    for i, word in enumerate(machine_words):
        print(f"{i:02}: {word:04X}")

    print("\n二进制比特流:")
    print(" ".join(f"{b:08b}" for b in binary))
    print("\nUART传输流:")
    print("".join(f"0{b:08b}1" for b in binary))

    

  \end{lstlisting}
\end{appendices}

\end{document}
