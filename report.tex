%!TEX program = xelatex
% 完整编译: xelatex -> biber/bibtex -> xelatex -> xelatex
\documentclass[lang=cn,a4paper,newtx]{elegantpaper}
\usepackage{algorithm}
\usepackage{algorithmicx}
\usepackage{algpseudocode}
\usepackage{subfig}
\usepackage{longtable}
\usepackage{lipsum}
% \usepackage[title]{appendix}
% \usepackage{subfigure}
% \usepackage{subfloat}
\renewcommand{\listtablename}{表格目录}
\renewcommand{\appendixname}{附录~\Alph{section}}
\title{计算机组织与结构II:CPU设计文档}
\author{李勃璘 \\ 吴健雄学院}

\version{1.0}
\date{\zhdate{2025/3/22}}

% 本文档命令
\usepackage{array}
\newcommand{\ccr}[1]{\makecell{{\color{#1}\rule{1cm}{1cm}}}}
\addbibresource[location=local]{reference.bib} % 参考文献,不要删除

\begin{document}

\maketitle
\thispagestyle{empty}
\begin{abstract}
本设计文档详细阐述了一款基于 五级流水线 的 CPU 体系结构及其 Verilog 实现。CPU 采用 取指(IF)、译码(ID)、执行(EX)、访存(MEM)、写回(WB) 五级流水线,以提高指令执行效率。文档首先介绍了 CPU 的总体架构,包括 时钟与复位机制、关键存储单元、内部数据通路与控制信号,并详细说明了 指令集架构 及其编码格式。随后,针对流水线执行过程中可能出现的 数据冒险、控制冒险、流水线暂停 等问题,提出了 旁路(Forwarding)、分支预测(Branch Prediction)、冒险检测(Hazard Detection) 等优化方案,并给出了相应的 Verilog 设计。最后,文档分析了 CPU 与内存、总线及外部控制信号 的交互方式,并探讨了 Verilog 在 FPGA 上的实现方案。该设计通过流水线优化与高效控制信号管理,提升了 CPU 的吞吐率,为后续硬件优化和扩展提供了良好的基础。
\end{abstract}
\vspace{1cm}

\textbf{版本更新记录:}

\begin{longtable*}{|c|c|p{10cm}|}
  \hline
  \textbf{版本号} & \textbf{日期} & \textbf{更新内容} \\
  \hline
  \endfirsthead

  \hline
  \textbf{版本号} & \textbf{日期} & \textbf{更新内容} \\
  \hline
  \endhead

  v1.0 & 2025-03-22 & 初始版本,包含基本 CPU 设计框架,流水线结构,Verilog 实现。 \\
  \hline
  v1.1 & 2025-03-25 & 修正部分时序问题,优化 Forwarding 和 Hazard Detection。 \\
  \hline
  v1.2 & 2025-03-28 & 增加 FPGA 实现部分,优化数据路径,改进分支预测机制。 \\
  \hline
  v1.3 & 2025-04-01 & 修正了一些公式错误,完善了性能分析和实验数据记录。 \\
  \hline

\end{longtable*}


\newpage
\pagenumbering{roman}
\tableofcontents
\newpage
\listoftables
\newpage
\pagenumbering{arabic}
\section{概述}
\lipsum[1]\footnote{手搓CPU是人类文明的伟大工程 —— 沃兹基硕德}


\section{体系结构设计}
\subsection{总体架构}
CPU由控制单元(CU),逻辑运算单元(ALU),内存(Memory)和寄存器组(Registers)组成,除内存以外,其余单元由被CU生成的控制信号控制的数据通路(Data Path)连接。另外,MAR和MBR分别还和地址总线、数据总线相连接,用于与内存交互。控制单元和内存都和控制总线相连接,用于与外部控制信号交互。

为简单起见,CPU的计算全部为\textbf{16位定点有符号数计算}。

\textcolor{blue}{这里需要一张图!!!}




\subsection{指令集架构}
指令集是指CPU能够对数据进行的所有操作的集合。每一条指令都可以被解释为寄存器与寄存器、内存、I/O端口之间的交互。交互方式由CU中的微指令(Micro-operation)给出,且每一条微指令都需要一个时钟执行(如不进行优化)。
\subsubsection{位宽设计}
地址段长为\textbf{8}位,指令码(Opcode)宽度为\textbf{8}位。因此,每一条指令的位宽为\textbf{16}位。
\subsubsection{寻址方式}
寻址方式指对地址段数据的解释方式。寻址方式由对应指令指定,支持表~\ref{tab:ISA:addressingmode}~中的全部寻址方式。由于给定的指令集高四位均空闲,使用最高位存储支持的寻址方式。
\begin{longtable}{c c c}
  \caption{指令集支持的寻址方式} \label{tab:ISA:addressingmode} \\
  \toprule
  寻址方式  & 描述 & 最高位\\
  \midrule
  \endfirsthead
  
  \caption[]{(续表)指令集支持的寻址方式} \\
  \toprule
  寻址方式  & 描述 & 最高位\\
  \midrule
  \endhead
  
  \midrule
  \multicolumn{3}{r}{续下页} \\
  \midrule
  \endfoot
  
  \bottomrule
  \endlastfoot
  
  立即数寻址   &  地址字段是操作数本身,数据为补码格式  & 0\\
  直接寻址 &  地址字段为存放操作数的地址    & 1\\
\end{longtable}

\subsubsection{指令集支持的指令}
指令集共支持13条不同的指令,列于表~\ref{tab:ISA:instructions}。每一条指令包含一个指令码,使用16进制格式存储。指令码的最高位为0时,寻址方式为立即数寻址;指令码的最高位为1时,寻址方式为直接寻址(此时HEX高位为8)。

\begin{longtable}{c c c}
  \caption{指令集包含指令及功能(立即数寻址下)} \label{tab:ISA:instructions} \\
  \toprule
  助记符  & 指令码(HEX) & 描述 \\
  \midrule
  \endfirsthead
  
  \caption[]{(续表)指令集包含指令及功能(立即数寻址下)} \\
  \toprule
  助记符  & 指令码(HEX) & 描述 \\
  \midrule
  \endhead
  
  \midrule
  \multicolumn{3}{r}{续下页} \\
  \midrule
  \endfoot
  
  \bottomrule
  \endlastfoot
  
  STORE X &  01   & [X] $\leftarrow$ ACC  \\
  LOAD X  & 02    & ACC $\leftarrow$ [X]  \\
  ADD X   & 03    & ACC $\leftarrow$ ACC + [X]\\
  SUB X   & 04    & ACC $\leftarrow$ ACC - [X]\\
  JGZ X   & 05    & ACC $\geq$ 0 ? PC $\leftarrow$ X : PC $\leftarrow$ PC + 1\\
  JMP X   & 06    & PC $\leftarrow$ X\\
  HALT    & 07    & Stop\\
  MPY X   & 08    & MR, ACC $\leftarrow$ ACC * [X],MR用于存储高16位 \\
  AND X   & 10    & ACC $\leftarrow$ ACC \& [X]\\
  OR X    & 11    & ACC $\leftarrow$ ACC | [X]\\
  NOT X   & 12    & ACC $\leftarrow$ \~[X] \\
  SHIFTR  & 13    & ACC $\leftarrow$ ACC $>>>$ 1,算术右移\\
  SHIFTL  & 14    & ACC $\leftarrow$ ACC $<<<$ 1,算术左移\\
\end{longtable}

\subsection{CPU内部寄存器}
该部分描述CPU内部寄存器的含义、存储格式和数据被解释为的格式。这些寄存器通过CPU的内部数据通路相连接。寄存器操作是CPU快速操作的核心。

\begin{longtable}{c c c c c c}
  \caption{CPU内部寄存器的含义、总存储条数、单位位宽和数据解释格式} \label{tab:CPU:datawidth} \\
  \toprule
  寄存器 & 含义 & 条数 & 位宽 & 数据解释格式 & 归属模块\\
  \midrule
  \endfirsthead

  \caption[]{(续表)CPU内部寄存器的含义、总存储条数、单位位宽和数据解释格式} \\
  \toprule
  寄存器 & 含义 & 条数 & 位宽 & 数据解释格式 & 归属模块\\
  \midrule
  \endhead

  \midrule
  \multicolumn{6}{r}{续下页} \\
  \midrule
  \endfoot

  \bottomrule
  \endlastfoot

  PC   & 程序计数器,存储当前指令地址             & 1  & 8   & 指令码(Opcode) & /\\
  MAR  & 内存地址寄存器,存储要访问的内存地址     & 1  & 8   & 地址码(Address)& /\\
  MBR  & 内存缓冲寄存器,存储从内存读取或写入的数据 & 1  & 16  & 二进制补码 & /\\
  IR   & 指令寄存器,存储当前正在执行的指令       & 1  & 8   & 指令码(Opcode)& /\\
  BR   & 通用寄存器,存储 ALU 计算中间结果        & 1  & 16  & 二进制补码 & ALU\\
  ACC  & 累加寄存器,存储 ALU 运算结果           & 1  & 16  & 二进制补码 & ALU\\
  MR   & 乘法寄存器,存储 ALU 乘法高 16 位       & 1  & 16  & 二进制补码 & ALU\\
  CM   & 控制存储器,存储微指令控制信号         & 未定 & 14  & 控制信号 & CU\\
  CAR  & 控制地址寄存器,指向当前执行的微指令   & 1  & 4   & CM中的条数下标 & CU\\
  CBR  & 控制缓冲寄存器,存储当前微指令的控制信号 & 1  & 14  & 控制信号 & CU\\
\end{longtable}

除上述寄存器以外,ALU进行运算时还会更改\textbf{状态寄存器}(Flags),用于CU进行条件判断。例如,JGZ命令需要判断上一步的运算结果是否是0,CU可以直接通过状态寄存器中的ZF(Zero Flag)寄存器直接进行判断。本设计中使用的所有状态寄存器见表~\ref{tab:CPU:status},它们都直接连向CU,通路不受控制信号的控制。

\begin{longtable}{c c c}
  \caption{状态寄存器列表} \label{tab:CPU:status} \\
  \toprule
  寄存器 & 全称 & 行为 \\ 
  \midrule
  \endfirsthead

  \caption[]{(续表)状态寄存器列表} \\
  \toprule
  寄存器 & 全称 & 行为\\
  \midrule
  \endhead

  \midrule
  \multicolumn{3}{r}{续下页} \\
  \midrule
  \endfoot

  \bottomrule
  \endlastfoot

  ZF   & Zero Flag             & ALU运算结果(通常为ACC)为0时置1\\
  CF  & Carry Flag     & 存储算术移位移出的比特(由于有符号数不存储进位)\\
  OF  & Overflow Flag &  ALU运算结果发生溢出时置1\\
  NF  & Negative Flag &  ALU运算结果为负数时置1\\
\end{longtable}

\subsection{CPU内部数据通路、控制信号与微操作指令(Micro-Operations)}
\subsubsection{数据通路与控制信号}
关键存储单元之间通过数据通路进行连接。每条数据通路都由一位控制信号控制。控制信号为1时表示通路打开,数据沿指定流向进行传输。该CPU共有\textbf{14}位控制信号。
\begin{longtable}{c c c}
  \caption{数据通路与控制信号一览} \label{tab:CPU:DataPath} \\
  \toprule
  控制信号位 & 源寄存器/单元  & 目的寄存器/单元  \\
  \midrule
  \endfirsthead

  \caption[]{(续表)数据通路与控制信号一览} \\
  \toprule
  控制信号位 & 源寄存器/单元  & 目的寄存器/单元  \\
  \midrule
  \endhead

  \midrule
  \multicolumn{3}{r}{续下页} \\
  \midrule
  \endfoot

  \bottomrule
  \endlastfoot

  0  & MAR   & 地址总线  \\
  1  & PC    & MBR  \\
  2  & PC    & MAR  \\
  3  & MBR   & PC  \\
  4  & MBR   & IR  \\
  5  & 数据总线 & MBR  \\
  6  & MBR   & ALU(待处理数据) \\
  7  & ACC   & ALU  \\
  8  & MBR   & MAR  \\
  9  & ALU   & ACC  \\
  10 & MBR   & ACC  \\
  11 & ACC   & MBR  \\
  12 & MBR   & 数据总线  \\
  13 & IR    & CU  \\
\end{longtable}

\subsubsection{微操作指令(Micro-Operations)}
为了实现指令集中所有指令,需要将指令集的指令分解为多步微操作指令。在本设计中,采用水平微指令(Horizontal Micro-operation)设计。
\subsection{内存(RAM)}
内存(RAM)存储指令集和CPU保存的数据。内存的大小为 512 Byte,每条存入内存的数据位宽为16,共能存入256条数据。其中,内存地址0到内存地址12预留为指令集。为了节省内存,不同寻址方式下的相同指令不占用两条内存,由CU改变指令的Opcode。

内存支持总线读写,并受三条总线控制:地址总线、数据总线和控制总线。\textbf{控制总线}中的外部控制信号决定在这个周期中内存的读/写状态,是否向数据总线写入,同步时序等功能。CPU与内存(RAM)通过两条总线交互,分别为\textbf{地址总线}和\textbf{数据总线}。内存通过读取地址总线决定写入内存中的地址,通过读取数据总线决定写入指定地址中的数据。关于总线的具体配置见~\ref{sec:ExternalControl}。

\subsection{总线与外部控制信号}\label{sec:ExternalControl}

\subsubsection{地址总线}
地址总线为\textbf{8}位单向总线,提供CPU(即MAR)到内存的地址传送通路。由于其为单向总线,仅需内存侧读使能信号与CPU侧MAR的控制信号控制即可,无需复用。
\subsubsection{数据总线}
数据总线为\textbf{16}位双向总线,提供CPU(即MBR)与内存的双向数据通路。数据总线采用分时复用的方式进行设计。\textcolor{red}{如果MEM和WB在同一时钟下,会出现流水线冲突,需要考虑是否引入旁路。}
\subsubsection{控制总线与外部控制信号}
外部控制信号是一组单比特信号,通过控制总线控制CU和RAM的行为,它们受流水线周期的控制置0或置1。CU和RAM通过内部映射决定监视哪些位的信号。外部控制信号主要包括以下功能:
\begin{itemize}
  \item RAM读写控制;
  \item 流水线控制信号。
\end{itemize}

所有外部控制信号列于表~\ref{tab:CPU:ExternalControl}。
\begin{longtable}{c c c c c}
  \caption{外部控制信号一览} \label{tab:CPU:ExternalControl} \\
  \toprule
  控制信号位/类型 & 别名  & 有效模块 & 高电平时作用 & 低电平时作用\\
  \midrule
  \endfirsthead

  \caption[]{(续表)外部控制信号一览} \\
  \toprule
  控制信号位/类型 & 别名  & 有效模块 & 高电平时作用 & 低电平时作用 \\
  \midrule
  \endhead

  \midrule
  \multicolumn{5}{r}{续下页} \\
  \midrule
  \endfoot

  \bottomrule
  \endlastfoot
  \textbf{RAM读写控制}\\
  \hline
  0  & MemoryWrite    & RAM  & RAM写数据总线 & 无\\
  1  & MemoryRead     & RAM  & RAM读数据总线和地址总线 & 无\\
  \hline
  \textbf{分支预测} \\
  \hline
  2   & BranchTaken    & CU   & 执行Branch     & 顺序执行 \\
  3   & Jump            & CU  & 执行Jump        & 顺序执行 \\
  \hline
  \textbf{流水线控制}  \\
  \hline
  4   & PipelineStall & CU    & 流水线暂停(如数据冒险、存储器访问延迟) & 无 \\
  5   & PipelineFlush & CU    & 流水线清空(如错误预测、异常发生)      & 无 \\

\end{longtable}
\subsection{流水线架构与优化策略}
\subsubsection{总体架构}
为了加速CPU的指令执行速度,采用\textbf{5级同步流水线}完成CPU执行指令的全流程。分别为:取指令(IF),指令译码(ID),指令执行(EX),内存访问(MEM)和写回(WB)五个阶段。流水线的五个阶段如下所示。
$$
  \text{IF} \rightarrow \text{ID} \rightarrow \text{EX} \rightarrow \text{MEM} \rightarrow \text{WB}
$$

对于本设计的指令集,仅LOAD和STORE指令涉及内存访问(MEM),STORE指令不需要写回(WB)阶段,因为内存写入在MEM阶段已经完成。
\subsubsection{流水线各阶段工作}

\subsubsection{流水线优化:分支预测}
在本设计中,采用1比特分支预测进行流水线优化。分支预测的控制信号由\textbf{控制总线}输入到CU中。分支预测的步骤如算法~\ref{alg:1bitBP}。

\begin{algorithm}[htbp]
  \caption{1-bit 分支预测}
  \label{alg:1bitBP}
  \begin{algorithmic}[1]
  \State \textbf{Note:} 使用 1-bit 预测器预测分支是否跳转
  \State \textbf{Input:} $PC, BranchTaken$
  \State \textbf{Output:} $NextPC$
  \State \textbf{Internal Registers:} $PredictorBit, BranchTarget$

  \Procedure{Branch Prediction}{}
      \If {$PredictorBit == 1$}
          \State $PredictedTaken \gets \text{True}$
          \State $NextPC \gets BranchTarget$
      \Else
          \State $PredictedTaken \gets \text{False}$
          \State $NextPC \gets PC + 2$
      \EndIf
  \EndProcedure

  \Procedure{Branch Resolution}{}
      \If {$BranchTaken \neq PredictedTaken$}  \Comment{预测错误}
          \State $Flush \gets 1$  \Comment{清空错误指令}
          \State $NextPC \gets \textbf{if } BranchTaken \textbf{ then } BranchTarget \textbf{ else } PC + 2$
      \EndIf
      \State \textbf{Update Predictor:}
      \State $PredictorBit \gets BranchTaken$  \Comment{更新 1-bit 预测器}
  \EndProcedure
  \end{algorithmic}
\end{algorithm}


% \subsubsection{流水线优化}\label{sec:streamline}
% 流水线设计下会出现内存读写冲突等,需要实现冲突现象的检测模块。同时,如果没有流水线优化策略,则每次访问冲突都需要清空流水线,这会降低流水线的运行效率。

\section{模块设计}
\subsection{时钟与复位}
CPU由\textbf{全局同步时钟}控制,所有控制逻辑与计算逻辑全部在时钟上升沿进行。CPU设有\textbf{全局异步复位}信号,低电平有效。当异步复位时,内存中除指令集数据以外所有数据清空,所有寄存器清空,控制信号全部归为断开(0)。
\subsection{ALU}

\subsection{CU}
参考表~\ref{tab:CPU:DataPath}~,确定指令集中的每一条指令对应的控制信号,并存储到CU的内部寄存器中,即可完成CU的主要功能设计。


\subsection{Memory}



% \subsection{分支预测}

\section{仿真验证}
\subsection{时延分析}
% 定义一个打印周期为从POC重置自身$SR_7$到一轮打印结束所经历的时钟上升沿个数。那么:
% \begin{enumerate}
%     \item 由POC重置到PROCESSOR置零$SR_7$,使用6个时钟,分别为读取POC信号的一个时钟和状态机的五个时钟;\footnote{中断模式较查询模式少用一个时钟。因为中断信号由组合逻辑生成,中断模式下PROCESSOR可立刻读取中断信号,无需读取POC信号。}
%     \item POC此时开始与打印机进行交互。写入\texttt{o\_tr}和\texttt{o\_pd}共使用1个时钟,\texttt{o\_tr}信号持续1个时钟;
%     \item 打印机收到信号后,耗时8个时钟打印数据;
%     \item 打印结束后,POC重置$SR_7$,使用1个时钟。
% \end{enumerate}
% 故一个打印周期为17个时钟(查询模式)或16个时钟(中断模式)。
\subsection{并行计算加速比(Speed-up Factor)分析}
\subsection{激励设置}

\section{FPGA实现}
\newpage
\addappheadtotoc
\begin{appendices}
  \section{完整设计代码}
\end{appendices}

\end{document}
